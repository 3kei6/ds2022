\documentclass[UTF8]{ctexart}

\usepackage{listings,xcolor}
\usepackage{graphicx}
\usepackage[a4paper,left=25.4mm,right=25.4mm,top=29.8mm,bottom=29.8mm]{geometry}

\lstset{numbers=left,
    commentstyle=\color{blue!50},
    columns=flexible}

\title{项目作业:五则运算计算器的实现}

\author{张竣凯 \\ 3210300361 \\ 数学与应用数学}

\begin{document}

\maketitle

A binary expression tree is a specific kind of a binary tree used to represent expressions. Two common types of expressions that a binary expression tree can represent are algebraic and boolean. These trees can represent expressions that contain both unary and binary operators.

\begin{flushright}
——摘自维基百科《Binary expression tree》
\end{flushright}

\section{设计思路}

\subsection{在 calculator.h 头文件中添加外置函数}

\hphantom 空1. dTOs(sTOd)函数 —— 将double转换成string(将string转换成double)\newline

\hphantom 空2. type函数 —— 为参数赋予指定类型,以方便处理后续任务\newline

\hphantom 空3. is\_valid函数 —— 判断式子是否符合规范

\subsection{在 calculator.h 头文件中添加 ExpressionTree 类}

\subsection{在 ExpressionTree 类的共有成员中添加必要的成员函数}
\hphantom 空1. 构造函数\newline

\hphantom 空2. 析构函数\newline

\hphantom 空3. insert函数 —— 自顶往下插入节点\newline

\hphantom 空4. clear函数 —— 清空ExpressionTree中的所有节点\newline

\hphantom 空5. calculate函数 —— 计算函数

\subsection{在 ExpressionTree 类的私有成员中添加必要的属性}

\hphantom 空1. ExpressionNode结构 —— 节点的结构,该结构中有着三名成员,分别为element和*left(*right),其中element表示节点的元素,*left(*right)表示指向左(右)子树的指针\newline

\hphantom 空2. priority函数 —— 判断运算符的优先级\newline

\hphantom 空3. insert函数 —— 自顶往下插入节点\newline

\hphantom 空4. clear函数 —— 清空ExpressionTree中的所有节点\newline

\hphantom 空5. calculate函数 —— 计算函数\newline

\hphantom 空6. result函数 —— 自底往上计算每个节点的值

\subsection{尽可能地使用引用\&以及充分考虑必要的const限制}
\hphantom 空以便于减少内部复制和提高安全性

\section{测试说明}

1. 在 main.cpp 中声明两个string类的变量eps和inp。\newline

2. 其中inp用于储存欲测试的式子。 \newline

3. 通过for循环来规范输入,并把合格的式子储存在eps中。\newline

4. 通过calculate函数来计算式子eps的值。\newline

5. 运行测试程序,\newline

当输入为 “2\^(1 + 3) - 5 *  (15:23)/(1 + 2) * 3 - 5” 时,终端出现以下结果:\newline

2\^(1 + 3) - 5 *  (15:23)/(1 + 2) * 3 - 5 = -65.15\newline

当输入为 “1.25 + (3 * (1 + 2\^2) * 3 - 43)\^(4 - 2)” 时,终端出现以下结果:\newline

1.25 + (3 * (1 + 2\^2) * 3 - 43)\^(4 - 2) = 5.25\newline

当输入为 “2\^(1 + 3)) - 5 *  (15:23)/(1 + 2) * 3 - 5” 时,终端出现以下结果:\newline

错误:括号不匹配\newline

当输入为 “2\^(1 + 3)) - 5 *  (15:23)/(1 - 1) * 3 - 5” 时,终端出现以下结果:\newline

错误:除数为0\newline

当输入为 “(1+2” 时, 终端出现以下结果:\newline

错误:无效的括号数量\newline

当输入为 “A+2” 时, 终端出现以下结果:\newline

错误:存在无效字符\newline

当输入为 “()1+2” 时, 终端出现以下结果:\newline

错误:无效的括号内容\newline

当输入为 “.1+2” 时, 终端出现以下结果:\newline

错误:无效的小数点位置\newline

当输入为 “12+” 时, 终端出现以下结果:\newline

错误:无效的运算符位置\newline

所有要求的参考算例都准确计算无误,在已完成参考算例的基础上,我选择补充我认为有必要的其他程序测试内容,且测试结果都正确无误。

\end{document}