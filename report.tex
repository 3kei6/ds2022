\documentclass[UTF8]{ctexart}

\usepackage{listings,xcolor}
\usepackage{graphicx}
\usepackage[a4paper,left=25.4mm,right=25.4mm,top=29.8mm,bottom=29.8mm]{geometry}

\lstset{numbers=left,
    commentstyle=\color{blue!50},
    columns=flexible}

\title{作业:图的模板类设计}

\author{张竣凯 \\ 3210300361 \\ 数学与应用数学}

\begin{document}

\maketitle

图是一种抽象数据类型,用于实现数学中图论的无向图和有向图的概念。图的数据结构包含一个有限(可能是可变的)的集合作为节点集合,以及一个无序对(对应无向图)或有序对(对应有向图)的集合作为边(有向图中也称作弧)的集合。节点可以是图结构的一部分,也可以是用整数下标或引用表示的外部实体。图的数据结构还可能包含和每条边相关联的数值(edge value),例如一个标号或一个数值(即权重,weight;表示花费、容量、长度等)。

\begin{flushright}
——摘自维基百科《图 (数据结构)》
\end{flushright}

\section{设计思路}

\subsection{在 Graph.h 头文件中添加 Graph 类}

\subsection{在 Graph 类的私有成员中添加必要的属性}

\hphantom 空1. Vertex结构 —— 节点的结构,该结构中有着两名成员,分别为value和neighbors。value为节点的值,neighbor为指定节点相邻的节点。\newline

\hphantom 空2. Edge结构 —— 边的结构,该结构中有着三名成员,分别为edge\_start、edge\_end、和weight。edge\_start和edge\_end分别为边的起点与终点,weight为边的权重。\newline

\hphantom 空3. graph\_type整型 —— 图的类型,取值范围为1和2,1表示无向无权图/有向无权图, 2表示有向无权图/有向有权图。\newline

\hphantom 空4. v数组和e数组 —— 分别存放指定图的所有节点和所有边。

\subsection{在 Graph 类的共有成员中添加必要的成员函数}
\hphantom 空1. 构造函数 —— 参数为图的类型graph\_type。\newline

\hphantom 空2. 析构函数 —— 无参数。\newline

\hphantom 空3. addVertex函数 —— 往指定图添加节点,参数为节点的值。\newline

\hphantom 空4. addNeighbors函数 —— 往指定图添加邻居,参数为节点的值、节点的邻居、以及它俩的权重。\newline

\hphantom 空5. listVertexes函数 —— 打印出指定图的所有节点。\newline

\hphantom 空6. listEdges函数 —— 打印出指定图的所有边。\newline

\hphantom 空7. listAdjList函数 —— 打印出指定图的邻接表。

\subsection{尽可能地使用引用\&以及充分考虑必要的const限制}
\hphantom 空以便于减少内部复制和提高安全性

\section{测试说明}

1. 在 main.cpp 中声明两个Graph类的变量g1和g2。其中g1为无权图,g2为有权图。\newline

2. 为图g1添加A、B、C和D四个节点。 \newline

3. 为图g1添加AB、AC、BD和DA四条边。\newline

4. 为图g2添加A、B、C和D四个节点。 \newline

5. 为图g2添加AB、AC、BD和DA四条边。\newline

6. 为图g2添加四条边的权重,分别为AB=3、AC=5、BD=4、DA=6。\newline

7. 最后也不忘使用打印出节点、边、邻接表的函数。\newline

8. 运行测试程序,在终端中出现了以下结果:\newline

\begin{flushleft}
测试无向无权图/有向无权图:\newline
列出全部节点: A, B, C, D.\newline

列出全部边: (A, B, 1), (A, C, 1), (B, D, 1), (D, A, 1).\newline

列出邻接表:\newline
A -> B, C, \newline
B -> D, \newline
C -> \newline
D -> A. \newline

测试结束\newline

测试无向有向图/有向有权图:\newline
列出全部节点: A, B, C, D.\newline

列出全部边: (A, B, 3), (A, C, 5), (B, D, 4), (D, A, 6).\newline

列出邻接表:\newline
A -> B(3), C(5), \newline
B -> D(4), \newline
C -> \newline
D -> A(6). \newline

测试结束\newline
\end{flushleft}

所有结果正确打印无误。

\end{document}
