\documentclass[UTF8]{ctexart}
\documentclass[a4paper]{article}

\usepackage{listings}
\usepackage{graphicx}
\usepackage[a4paper,left=25.4mm,right=25.4mm,top=29.8mm,bottom=29.8mm]{geometry}

\lstset{numbers=left,
	basicstyle=\ttfamily,
	language=c++}

\title{作业一: 双链表的设计和实现}

\author{张竣凯 \\ 3210300361 \\ 数学与应用数学}

\begin{document}

\maketitle

双向链表也叫双链表,是链表的一种,它的每个数据结点中都有两个指针,分别指向直接后继和直接前驱。所以,从双向链表中的任意一个结点开始,都可以很方便地访问它的前驱结点和后继结点。一般我们都构造双向循环链表。

\begin{flushright}
——摘自百度百科《双向链表》
\end{flushright}




\section{设计思路}

\subsection{在 DoubleLinkedList.h 头文件里头编写双链表的类}
\hphantom 空即:class DoubleLinkedList

\subsection{编写该类中所需的变量、结构、子类、以及基本功能等}
\hphantom 空例如:Node结构、Iterator类、begin函数等

\subsection{编写出结构所需的变量}
\hphantom 空例如:Node结构中包含数据以及两个指针

\subsection{编写出各个子类所需的函数}
\hphantom 空例如:Iterator类中的各种运算符重载函数等

\subsection{编写出双链表类所需的函数}
\hphantom 空例如:begin函数、end函数、size函数等

\subsection{在 main.cpp 主流程里头编写外部函数find以及测试程序}
\hphantom 空尽可能地使用引用&以减少内部复制,以及充分考虑必要的const限制以提高安全性

\newpage

\section{额外函数}

\subsection{printList函数}

\begin{lstlisting}
void printList() 
{
    for (iterator itr = begin(); itr != end(); itr++)
    {
        std::cout << *itr << " ";
    }
    std::cout << '\b' << std::endl;
}
\end{lstlisting}

\subsection{find函数}

\begin{lstlisting}
template <typename DT>
typename DoubleLinkedList<DT>::iterator find
        (DoubleLinkedList<DT>& _list, const DT& _val)
{
    typename DoubleLinkedList<DT>::iterator itr = _list.begin();
    typename DoubleLinkedList<DT>::iterator end = _list.end();
    int position = 1;
    for(itr; itr != end; ++itr)
    {
        if(*itr == _val)
        {
            std::cout << "The position of value \"" << _val << "\" is " 
                      << position << "." << std::endl;
            return itr;
        }
        position++;
    }
    std::cout << "The given value \"" << _val 
              << "\" is not found in the list!" << std::endl;
    return itr;
};
\end{lstlisting}

\newpage

\subsection{main函数}

\begin{lstlisting}
int main(int argc, char* argv[])
{
    DoubleLinkedList<int> L;
    L.push_back(1);
    L.push_back(2);
    L.push_back(3);
    L.push_back(4);
    L.push_back(5);
    L.printList();
    DoubleLinkedList<int>::iterator iter;
    iter = find(L, 3);
    L.erase(iter);
    L.printList();
    iter = find(L, 3);
    return 0;
}
\end{lstlisting}

\section{测试说明与结果}

将main.cpp与DoubleLinkedList.h两个文件放入项目list后,进入终端,找到list的路径后,输入g++ -o test main.cpp后回车,接着输入./test后回车,将会出现以下结果:

\begin{flushleft}
1 2 3 4 5 \\
The position of value "3" is 3. \\
1 2 4 5 \\
The given value "3" is not found in the list! \\
\end{flushleft}

\section{内存泄漏检查}

承接上个部分,继续在终端中输入valgrind ./test后回车,将会出现以下结果

\begin{flushleft}
==2878== \\
==2878== HEAP SUMMARY: \\
==2878==     in use at exit: 0 bytes in 0 blocks \\
==2878==   total heap usage: 9 allocs, 9 frees, 73,896 bytes allocated \\
==2878== \\
==2878== All heap blocks were freed -- no leaks are possible \\
==2878== \\
==2878== For lists of detected and suppressed errors, rerun with: -s \\
==2878== ERROR SUMMARY: 0 errors from 0 contexts (suppressed: 0 from 0) \\
\end{flushleft}

\end{document}
